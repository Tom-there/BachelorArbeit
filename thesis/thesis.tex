\documentclass{hhuthesis}

\usepackage[english,ngerman]{babel} % Deutsch

\author{Tom Schreiner}
\title{Visualisieren von Algorithmen in Compilern}
\gratuationtype{Bachelor}

%% Beginn- und Abgabedaten der Arbeit
\begindate{05. November 2024} % Beginn
\duedate{05. Februar 2025} % Abgabe

%% Erst- und Zweitgutachter
\firstexaminer{John~Witulski}
\secondexaminer{Fabian~Ruhland}

\usepackage[textsize=scriptsize]{todonotes}

%% Zeige Zeilennummern in der Arbeit an.
%% Der \linenumbers Befehl muss hierzu aufgerufen werden.
%% Praktisch für Feedback Ihrer potentiellen Korrekturleser!
\usepackage{lineno}
% \linenumbers % <- Kommentar entfernen!


%% Häufig benutzte mathematische Packages.
\usepackage{amsfonts}
\usepackage{amsmath}
\usepackage{amssymb}


\usepackage{listings} % Einbindung von Code
\usepackage{algorithmicx} % Angabe von Algorithmen in Pseudocode
\usepackage{siunitx} % \num Befehl zum einfacheren Formatieren von Zahlen.
\usepackage{enumitem} % Leichter konfigurierbare enumerate-Umgebungen.
\usepackage{subcaption} % Unterteilung von Figures in Subfigures.
\usepackage{hyperref} % Klickbare Referenzen (z.B. im Inhaltsverzeichnis)
\usepackage{url} % \url Kommando für Darstellung von Links
\usepackage{csquotes} % Improved quoting.
\usepackage{xspace} % Nicht terminierte Kommandos essen keinen Whitespace mehr.

%% Tabellen
\usepackage{tabularx} % tabularx Umgebung für mehr Kontrolle über Tabellen.
\usepackage{booktabs} % \toprule, \midrule, \bottomrule
\usepackage{multirow}
\usepackage{multicol}
\usepackage{longtable} % Große Tabellen gehen über mehrere Seiten.

%% Intelligenteres Referenzieren mittels \cref.
%% \languagename um dynamisch zwischen ngerman oder english zu wechseln.
\usepackage[\languagename,capitalize]{cleveref}

%%%%%%%%%%%%%%%%%%%%%%%%%%%%%%%%%%%%%%%%%%%%%%%%%%%%%%%%%%%%%%%%%%%%%%%%%%%%%%%%
%% (Ende) LaTeX Packages in Nutzung                                           %%
%%%%%%%%%%%%%%%%%%%%%%%%%%%%%%%%%%%%%%%%%%%%%%%%%%%%%%%%%%%%%%%%%%%%%%%%%%%%%%%%


\begin{document}
%% Set up title page, declaration of authorship, abstract, acknowledgements
\frontmatter
\makefrontmatter
\tableofcontents

\mainmatter

%%%%%%%%%%%%%%%%%%%%%%%%%%%%%%%%%%%%%%%%%%%%%%%%%%%%%%%%%%%%%%%%%%%%%%%%%%%%%%%%
%% Der Inhalt der Arbeit                                                      %%
%%                                                                            %%
%% Hier können Sie die schriftliche Ausarbeitung ihrer Arbeit                 %%
%% niederschreiben. Der Übersicht halber bietet sich jedoch an, dies in einer %%
%% oder mehreren separaten Dateien zu tun, welche mittels \input eingebunden  %%
%% werden --- wie auch in der Vorlage geschieht.                              %%
%%%%%%%%%%%%%%%%%%%%%%%%%%%%%%%%%%%%%%%%%%%%%%%%%%%%%%%%%%%%%%%%%%%%%%%%%%%%%%%%

\input{tex/einleitung.tex}
\newpage
\input{tex/implementierung.tex}
\newpage
\input{tex/einordnung.tex}

%% Dieser Part kann auskommentiert werden, sollte kein Anhang nötig sein
% \appendix
% %!TeX root: thesis.tex
\section{Zusätzliche Informationen}

Hier können Sie Ihren Anhang definieren.

Achten Sie darauf, dass der Anhang in Ihrer \texttt{thesis.tex}
initial auskommentiert ist.
Der entsprechende Part befindet sich nahe dem Ende der Datei.
Entfernen Sie bei Bedarf die Kommentierung um den Anhang nutzen zu können.


\section{Nutzung}

Der Anhang wird wie die Abschnitte des Hauptteils der Arbeit gestaltet,
also mit \texttt{\textbackslash section} Befehlen.

  \subsection{Unterabschnitte}
  Die Verwendung von Unterabschnitten im Anhang
  mittels \texttt{\textbackslash subsection}
  funktioniert ebenfalls!


%%%%%%%%%%%%%%%%%%%%%%%%%%%%%%%%%%%%%%%%%%%%%%%%%%%%%%%%%%%%%%%%%%%%%%%%%%%%%%%%
%% (Ende) Der Inhalt der Arbeit                                               %%
%%%%%%%%%%%%%%%%%%%%%%%%%%%%%%%%%%%%%%%%%%%%%%%%%%%%%%%%%%%%%%%%%%%%%%%%%%%%%%%%

\backmatter
\listoffigures
\listoftables

\clearpage
\bibliography{references}
%% Depending on Language, use german alphadin or original alpha
\iflanguage{ngerman}{
  \bibliographystyle{alphadin}
}{
  \bibliographystyle{alpha}
}

\end{document}
