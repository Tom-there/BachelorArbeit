%!TeX root: thesis.tex
Das Ziel dieser Bachelorarbeit war es,
einige Algorithmen und Konzepte
aus dem Compilerbau zu visualisieren
um Personen, insbesondere Studenten,
eine Hilfe beim Studieren dieser zu geben.

Dafür wurde ein Framework entwickelt, welches
diese und beliebige weitere Algorithmen verständlich
und Schritt für Schritt visualisieren kann.

Folgende Algorithmen wurden implementiert:
\begin{enumerate}
  \item Analyse von erreichenden Definitionen für Grundblöcke 
  \item Liveness Analyse für Grundblöcke 
  \item Liveness Analyse für einzelne 3-Address-Code Instruktionen
  \item Erstellen von Grundblöcken für ein Programm geschrieben in 3-Address-Code
  \item Erstellen eines Kontrollflussgraphen für ein 3-Address-Code Programm
\end{enumerate}

Um diese Algorithmen simpel und gut verständlich zu visualisieren, 
brauchte das Framework zwei Arten von Darstellungen:
\begin{enumerate}
  \item Graphen: um Kontrollflussgraphen darzustellen
  \item Tabellen: um Datenflusswerte darzustellen
    und 3-Address-Code in einer angenehmen Art und Weise zu visualisieren
\end{enumerate}
Desweiteren brauchte es die Möglichkeit, 3-Address-Code
und Grundblöcke einfach zu verarbeiten.


Außerdem ist es durch Implementierung eines Interfaces einfach, 
weitere Algorithmen hinzuzufügen.
Es wurden weitere Interfaces implementiert, mit 
denen häufig genutzte Funktionalitäten, wie zum Beispiel 
ein Button, um Code aus einer Datei zu laden, einfach in neuen
Plugins genutzt werden können. Somit können weitere Plugins
mit sehr viel weniger Aufwand hinzugefügt werden.
