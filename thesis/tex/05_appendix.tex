%!TeX root: thesis.tex
\section{Codebeispiele}
\begin{lstlisting}[language=Java, caption={Generieren der ID für eine Kante}, label={cde:edge-id}]
public String getID(){
  return "("+sourceNode.getId()+") -> ("+targetNode.getId()+")";
}
\end{lstlisting}

\begin{lstlisting}[language=Java, caption={Konstruktor der Klasse ThreeAddressCodeInstruction}, label={cde:graph-constructor}]
public Graph(Location location){
  this.location = location;
  exportedPanel = new JPanel(new BorderLayout());
  switch(location){...} // Groesse des exportierten JPanels festlegen
  
  nodes = new HashSet<>();
  edges = new HashSet<>();
  
  graph = new MultiGraph("Graph");
  graph.setAttribute("ui.stylesheet", ...); // stylesheet festlegen
  viewer = new SwingViewer(graph, ...); // threading festlegen
  View view = viewer.addDefaultView(false);

  layout = new LinLog();
  viewer.enableAutoLayout(layout);

  exportedPanel.add((Component) view, BorderLayout.CENTER);
}
\end{lstlisting}
\begin{lstlisting}[language=Java, caption={Implementierung der addNode(Node) Methode}, label={cde:graph-addNode}]
  public void addNode(Node newNode){
    if(nodes.contains(newNode))
      return;
    nodes.add(newNode);
    graph.addNode(newNode.getId());
    layout.shake();
  }
\end{lstlisting}
\begin{lstlisting}[language=Java, caption={Implementierung der setLabelOfNode(Node, String) Methode}, label={cde:graph-labelNode}]
  public void setLabelOfNode(Node Node, String label){
    if(nodes.contains(newNode))
      graph.getNode(node).setAttribute("ui.label", label);
  }
\end{lstlisting}

\begin{lstlisting}[language=Java, caption={Implementierung der setValueAtMethode(Object, int, int) Methode}, label={cde:setValueAt}]
  public void setValueAt(Object value, int rowIndex, int colIndex){
    if(value == null)
      value = "";
    try{
      data[rowIndex][colIndex] = value.toString();
      TableModelEvent event = new TableModelEvent(...);
      listeners.forEach(l->l.tableChanged(event));
    }catch(ArrayIndexException e){
      // Error handling
    }
  }
\end{lstlisting}


\begin{lstlisting}[language=Java, caption={Implementierung der resizeTable(int, int) Methode}, label={cde:resizeTable}]
public void resizeTable(int rows, int cols) {
  if(rows==tableModel.getRowCount() 
  && cols == tableModel.getColumnCount())
    return;
  tableModel = new DataTableModel(rows, cols);
  this.setModel(tableModel);
}
\end{lstlisting}

\begin{lstlisting}[language=Java, caption={Implementierung der highlightLine(int) Methode}, label={cde:highlightLine}]
public void highlightLine(int line) {
  this.clearSelection();
  if (line == 0)
    return;
  try {
    this.addRowSelectionInterval(line, line);
  }catch (IllegalArgumentException e){
    ... //Handle wrong lines error
  }
    }
}
\end{lstlisting}

\begin{lstlisting}[language=Java, caption={Implementierung der getToolTipText(MouseEvent) Methode}, label={cde:tooltip}]
  public String getToolTipText(MouseEvent mouseEvent) {
    String tip = null;
    Point p = mouseEvent.getPoint();

    int rowIndex = rowAtPoint(p);
    int colIndex = columnAtPoint(p);

    try{
      if(rowIndex<getRowCount() && rowIndex> -1
      && colIndex<getColumnCount() && colIndex> -1)
        tip = tableModel.getValueAt(rowIndex, colIndex).toString();
    }catch (NullPointerException ignored){}
    
    return tip;
  }
\end{lstlisting}

\begin{lstlisting}[language=java, caption={Implementierung der resizeColumnDisplay() Methode}, label={cde:resize}]
  public void resizeColumnDisplay() {
    TableColumnModel columnModel = this.getColumnModel();
    for (int i = 0; i < columnModel.getColumnCount()-1; i++) {
      int width = 0;
      for (int j = 0; j < tableModel.getRowCount(); j++) {
        TableCellRenderer renderer = this.getCellRenderer(j, i);
        Component component = this.prepareRenderer(renderer, j, i);
        width = Math.max(component.getPreferredSize().width+2, width);
      }
      columnModel.getColumn(i).setMinWidth(width);
      columnModel.getColumn(i).setMaxWidth(width);
    }
    int width = 10;
    for (int j = 0; j < tableModel.getRowCount(); j++) {
      TableCellRenderer renderer = this.getCellRenderer(j, columnModel.getColumnCount()-1);
      Component component = this.prepareRenderer(renderer, j, columnModel.getColumnCount()-1);
      width = Math.max(component.getPreferredSize().width+2, width);
    }
    columnModel.getColumn(columnModel.getColumnCount()-1).setMinWidth(width);
    columnModel.getColumn(columnModel.getColumnCount()-1).setMaxWidth(this.getMaximumSize().width);
  }
\end{lstlisting}




\begin{lstlisting}[language=java, caption={Konstruktor der Klasse ThreeAddressCodeInstruction}, label={cde:taci-constructor}]
public ThreeAddressCodeInstruction(String rawInput, int address){
  ...
  String[] pieces = rawInput.split(" ");
  switch (pieces.length) {
    case 2 -> ... //Unbedingter Sprung goto X
    case 3 -> ... //Kopierbefehl X = Y
    case 4 -> { //entweder unaere Operation oder ein bedingter Sprung
      switch(pieces[2]){
        case "-" -> ... //Unaere Operation X = - Y
        case "goto" -> {//ein bedingter Sprung
          switch(pieces[0]){
            case "if" -> ... //bedingter Sprung if Y goto X
            case "ifFalse" -> ... //bedingter Sprung ifFalse Y goto X
          }
        }
      }
    case 5 -> ... //Binaere Operation X = Y op Z
    case 6 -> { //bedingter Sprung
      switch(pieces[2]){
        case "<" -> ...//if Y < Z goto X
        case ">" -> ...//if Y > Z goto X
        case "<=" -> ...//if Y <= Z goto X
        case ">=" -> ...//if Y >= Z goto X
        case "==" -> ...//if Y == Z goto X
        case "!=" -> ...//if Y != Z goto X
      }
    }
    ...
 }
\end{lstlisting}

\begin{lstlisting}[language=Java, caption={Der Konstruktor der ThreeAddressCode Klasse}, label={cde:tac-constructor}]
public ThreeAddressCode(String raw){
  List<String> inputLines = raw.lines().toList();
  code = new ArrayList<>(inputLines.size());
  for (int i = 0; i < inputLines.size(); i++) {
    code.add(new ThreeAddressCodeInstruction(inputLines.get(i), i));
  }
  basicBlocks = toBBList(code);
}
\end{lstlisting}

\begin{lstlisting}[language=Java, caption={Zusammenfassung der toBBList() Methode}, label={cde:bb-gen}]
private static List<BasicBlock> toBBList(List<...> code){
  Set<ThreeAddressCodeInstruction> leaders = new HashSet<>(1);
  if(!code.isEmpty()) //die erste Instruktion ist immer Leader
    leaders.add(code.get(0));
  for (int i = 0; i < code.size(); i++) {
    ... // Finde alle Leader 
  }
  List<ThreeAddressCodeInstruction> sortedLeaders = leaders.stream().sorted(Comparator.comparingInt(ThreeAddressCodeInstruction::getAddress)).toList();
  
  // Finde die letzte Instruktion jedes Grundblockes
  List<Integer> firstAddresses = sortedLeaders.stream()
                .map(ThreeAddressCodeInstruction::getAddress)
                .toList();
  List<Integer> lastAddresses = new ArrayList<>(leaders.size());
  for (int i = 1; i < sortedLeaders.size(); i++) {
    lastAddresses.add(firstAddresses.get(i)-1);
  }
  lastAddresses.add(code.getLast().getAddress());
  
  List<List<Integer>> fAOS = new ArrayList<>(leaders.size());
  for (int lastAddress : lastAddresses) {
    ... // Bestimme die Addressen der Nachfolger
  }
 

  // Setze die Listen zu einer BasicBlock Liste zusammen
  List<BasicBlock> basicBlocks= new ArrayList<>(leaders.size());
  for (int i = 0; i < firstAddresses.size(); i++) {
    int firstAddress = firstAddresses.get(i);
    int lastAddress = lastAddresses.get(i);
    List<Integer> successors = fAOS.get(i);
    BasicBlock basicBlock = new BasicBlock(
      firstAddress, lastAddress, successors
    );
    basicBlocks.add(basicBlock);
  }
  return basicBlocks;
}
\end{lstlisting}
\newpage

\begin{lstlisting}[language=Java, caption={Ausschnitt der toBBList()-Methode zum finden aller Leader}, label={cde:bb-leader}]
for (int i = 0; i < code.size(); i++) {
  if(code.get(i).canJump()){
    ThreeAddressCodeInstruction destination = code.get(
      Integer.parseInt(code.get(i).getDestination())
    );
    leaders.add(destination);
    if(i+1<code.size()) 
      leaders.add(code.get(i+1));
} }
\end{lstlisting}

\begin{lstlisting}[language=Java, caption={Zusammenfassung der toBBList() Methode}, label={cde:bb-successor}]
List<List<Integer>> fAOS = new ArrayList<>(leaders.size());
for (int lastAddress : lastAddresses) {
  List<Integer> successors = new ArrayList<>(0);
  fAOS.add(successors);
  switch (code.get(lastAddress).getOperation()) {
    case jmp -> 
      successors.add(
        Integer.valueOf(code.get(lastAddress).getDestination())
      );
    case booleanJump, negatedBooleanJump, 
         eqJump, geJump, gtJump, leJump, ltJump, neJump -> {
      successors.add(
        Integer.valueOf(code.get(lastAddress).getDestination())
      );
      if (lastAddress+1 < code.size())
        successors.add(lastAddress + 1);
    }
    default -> {
      if (lastAddress+1 < code.size())
        successors.add(lastAddress + 1);
} } }
\end{lstlisting}
\begin{lstlisting}[language=Java, caption={Bestimmmen der gen Menge für erreichende Definitionen}, label={cde:rd_gen}]
for(BasicBlock block:basicBlocks){
  Set<Integer> currentGenSet = gen.get(block);
  for (int i = block.lastAddress(); i >= block.firstAddress(); i--) {
    ThreeAddressCodeInstruction currentInstruction = code.get(i);
    if(!currentInstruction.writesValue())
      continue;
    boolean newVar = true;
    for (int j : currentGenSet) {
      if(currentInstruction.getDestination().equals(code.get(j).getDestination()))
        newVar = false;
    }
    if(newVar)
      currentGenSet.add(i); 
  }
}
\end{lstlisting}

\begin{lstlisting}[language=Java, caption={Bestimmmen der kill Menge für erreichende Definitionen}, label={cde:rd_kill}]
for(BasicBlock block:basicBlocks){
  Set<Integer> currentKillSet = kill.get(block);
  for (Integer i:currentGenSet) {
    for (int j = 0; j < code.size(); j++) {
      if(!code.get(j).writesValue())
        continue;
      if(j == i)
        continue;
      if(code.get(j).getDestination().equals(code.get(i).getDestination()))
        currentKillSet.add(j);
    }
  }
}
\end{lstlisting}

\begin{lstlisting}[language=Java, caption={Bestimmmen der Menge der Vorfahren für erreichende Definitionen}, label={cde:rd_anc}]
for (BasicBlock block:basicBlocks) {
  List<Integer> firstAddressesOfSuccessors = block.firstAddressesOfSuccessors();
  for (int address:firstAddressesOfSuccessors) {
    for (BasicBlock potentialSuccessor:basicBlocks) {
      if(potentialSuccessor.firstAddress() == address)
        ancestors.get(potentialSuccessor).add(block);
    }
  }
}
\end{lstlisting}
