%! TeX root: thesis.tex
\section{Verwandte Arbeiten}

Wie im \cref{motiv} bereits erwähnt, gibt es einige andere Arbeiten,
welche ähnliche Aufgaben haben:\\

Eine davon ist die STUPSToolbox\cite{toolbox} programmiert von Fabian Ruhland
\footnote{Warum kommt mir der Name so bekannt vor? :)}
und erweitert von Isabel Wingen.\\
Diese Toolbox enthält einige Funktionaliäten für die Arbeit mit Automaten
und Grammatiken. Ursprünglich war die Idee diese Toolbox weiter zu entwickeln,
jedoch funktioniert diese leider mittlerweile nicht mehr ohne weiteres. Daher wurde
sich dagegen entschieden diese Toolbox zu erweitern oder instand zu setzen.\\

Ein Tool mit dem auch liveness Analyse gemacht werden kann ist der
Data Flow Analysis Visualizer\cite{dfav}. Dies ist eine Abschlussarbeit\cite{dfavpres}
eines Compilerdesign Kurses der UCSD
\footnote{University of California San Diego}.
Leider kann in diesem Programm nur JavaScript-Code eingegeben werden
und auch nur eine Liveness Analyse auf einzelnen Instruktionen ausgeführt werden.

Letztlich gibt es noch ein Tool namens VisOpt\cite{VisOpt}.
In diesem kann man Jova\footnote{Eine Subsprache von Java}-Code eingeben,
ihn in 3-Address-Code übersetzen lassen und sich eine Vielzahl 
von Optimierungsmöglichkeiten visualisieren lassen.\\
Leider ist es jedoch nicht möglich direkt 3-Address-Code einzugeben,
sich Grundblöcke anzeigen zu lassen oder eine reaching Definitions Analyse
durchzuführen.\\

\newpage
\section{Ausblick}
Das in dieser Arbeit entwickelte Framework bietet ein Fundament
um viele Analyse- und Optimierungsalgorithmen zu visualisieren.\\
Leider ist es im vorgegebenen Zeitrahmen nicht möglich gewesen mehr als
die in der Arbeit besprochenen Plugins zu schreiben.\\

Folgende Algorithmen können ohne Erweiterung des Frameworks implementiert werden:
\begin{itemize}
  \item Konstantenpropagation
  \item Verfügbare Ausdrücke
  \item Eliminierung von Redundanz
\end{itemize}

Durch Implementierung von Automaten im Framework können ausserdem weitere 
Konzepte visualisiert werden die Studenten des Compilerbaus helfen könnten:
\begin{itemize}
  \item Erstellen von Lexern für die Tokenisierung
    \begin{enumerate}
      \item nicht-determinischer endlicher Automat (NEA)
      \item deterministischer endlicher Automat (DEA)
      \item minimierter DEA
    \end{enumerate}
    Ausserdem auch durchführung einer Tokenisierung von Quellcode
  \item Erstellen von Parsern für den Bau von Syntaxbäumen
    \begin{enumerate}
      \item Top-Down parsing
      \item Bottom-Up parsing
    \end{enumerate}
  \item Generierung von 3-Address-Code aus einem Syntaxbaum
\end{itemize}



\newpage
\section{Herausforderungen und Fazit}
Das im Exposee vereinbarte Ziel dieser Bachelorarbeit war es folgende Konzepte zu visualisieren:
\begin{itemize}
  \item Das Bauen eines Kontrollflussgraphen aus 3-Addres-Code
  \item Das Visualisieren von Constant Folding
  \item Das Visualisieren von liveness Analysen
  \item Und das Visualisieren von reaching Definitions
\end{itemize}

Leider war es aufgrund von falsch eingeschätztem Arbeitsaufwand nicht möglich 
den constant Folding Algorithmus zu implementieren.\\

Das Implementieren des Frameworks war hingegen sehr erfolgreich.\\
Es war möglich ein minimales Interface zu erstellen, welches alle
Funktionen des Frameworks abbilden kann. Mit diesem können neue Plugins
mit sehr geringem Aufwand hinzugefügt werden.\\

Ausserdem wurde das Framework samt Plugins einer Gruppe 
Student*innen des Compilerbau Moduls vorgestellt.
Das Feedback der Studierenden war positiv, die vorgestellten Plugins
seien nützlich um zum Beispiel Übungsaufgaben zu bearbeiten,
da die Lösung Schritt für Schritt abgleichbar ist.\\
